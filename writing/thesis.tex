\documentclass[12pt]{article}

\usepackage{amsmath, amssymb}
\usepackage{graphicx}
\usepackage[margin=1in]{geometry}
\usepackage[style=apa]{biblatex}
\usepackage{booktabs}
\usepackage{rotating}
\usepackage{wrapfig}
\usepackage{caption}
\usepackage{longtable}
\usepackage{setspace}
\usepackage[flushleft]{threeparttable}
\usepackage{siunitx}
\usepackage[hidelinks]{hyperref}
\geometry{a4paper, margin=1in}
\doublespacing{}
\addbibresource{references.bib}
\graphicspath{ {../images/output/} }

\sisetup{
  input-symbols = (),
  table-align-text-post = false,
  table-text-alignment = center,
%   parse-numbers = true
}

% \title{}
\author{Your Name}
\date{Date of submission: \today}
\renewcommand{\arraystretch}{1.3}
% -------------------------------------------------------------------------------------------
\begin{document}

\begin{titlepage}
    \begin{center}
        \vspace*{1cm}
 
        \Large
        \textbf{Does Greenwashing Pay Off?}
 
        \vspace{0.5cm}
        An Empirical Analysis of Cheap-Talk in Firm Behaviour.
        
        \vspace{1.5cm}
        
        \textbf{Jakub Przewoski}
        
        \vspace{0.5cm}
        \large
        Administrative Number:\@ u431929\\ 
        Student Number:\@ 2092491 \\
        Bachelor of Science in Economics \\
        
        \vfill
        
        Tilburg University\\
        Department of Economics\\
        Supervisor: David Schindler\\
        % \includegraphics[width=0.5\textwidth]{../images/TiU logo_kleur_transparant.png}
        
        \vspace{0.8cm}
        
        % \large
        Date of Submission: \today{} \\
        Number of words: 7542

            
    \end{center}
\end{titlepage}

\pagebreak
\pagenumbering{roman}

% \begin{abstract}
% Your abstract goes here.


% \end{abstract}
% \pagebreak

\tableofcontents
\pagebreak
\listoffigures
\listoftables
\pagebreak

% ------------------------------ actual thesis starts ---------------------
\pagenumbering{arabic}
\setcounter{page}{1}

\section{Introduction}\label{sect:introduction}

In recent years, climate change has become more pronounced in the public discourse, due to its global importance. Policymakers, international organizations as well as consumers have been growing more aware of the problems facing the world in the upcoming years \parencite{pew2022climate}. This awareness has lead to a rapid expansion of the sustainable finance industry, with as much as \$20.6 billion dollars flowing into ESG investment funds in 2019 \parencite{hale2020sustainable}. At the same time, companies trying to answer the needs of their consumers have reoriented their supply chains and started transitioning to more sustainable methods of production \parencite{pwc2023supplychain, pwc2023sustainability}. However, while this structural shift can bring about benefits, it also opens up avenues for greenwashing \parencite{villenaMoreSustainableSupply2020}. 

According to \textcite{lyonMeansEndGreenwash2015}, greenwashing is defined as ``any communication that misleads people into adopting overly positive beliefs about an organization's  environmental performance, practices, or products'', and is generally regarded as a phenomenon that has negative effects on the performance of the markets. For example, both theoretical \parencite{wuBadGreenwashingGood2020, cartellierCanInvestorsCurb2023} and empirical \parencite{barrageAdvertisingEnvironmentalStewardship2020, marquisScrutinyNormsSelective2016, kimGreenwashVsBrownwash2015} literature suggest, that when investments in sustainability are largely unobservable, it may be optimal for a firm to appear sustainable through observable actions, such as corporate communication, while not carrying through with their investment. Engaging in such behaviour can imply a misallocation of capital, as shareholders would invest their funds in companies that do not align with their sustainability goals \parencite{kimGreenwashVsBrownwash2015}, and can cause a decrease in the welfare of consumers, as misleading marketing may imply purchases of products that are not aligned with their sustainable preferences \parencite{barrageAdvertisingEnvironmentalStewardship2020}.

% work more on the second sentence!
At the same time, relatively little is known about the profitability of greenwashing, when demand for sustainability suddenly rises. It remains unclear, what are the consumer and investor reactions following, for example, an increase in interest around climate change, and what are their implications for the financial performance of greenwashing firms. For example, if these stakeholders can identify firms that engage in greenwashing, they could switch away from these companies, when they start demanding more sustainable action, which could have negative implications for companies' financial performance. Alternatively, if they were to believe that a company is credible in their efforts, a demand shock could be profitable for greenwashing firms as they would appear more sustainable.  


My approach in investigating this issue consists of three steps. Firstly, I collect a large sample of sustainability reports for a set of publicly listed companies. After extracting their contents, the reports are used to create a Cheap-Talk-Index ($CTI$) that gauges the credibility of climate commitments made by these companies \parencite{binglerHowCheapTalk2024}. This index serves as a proxy for perceived greenwashing, following the contributions of~\cite{coenAreCorporateClimate2022}.

Secondly, I exploit the positive sustainability demand shock in the form of the world-wide September 2019 climate strikes organized by Fridays For Future (20 \& 27 September 2019) during the ``Global Week for Future''. As these events attracted a lot of attention (which is further explored in the next section), they provide a natural setting to study the effect of perceived greenwashing on the financial performance of companies. Using a multivalued difference-in-differences (DiD) design, I test whether higher greenwashing intensity had an impact on the Cumulative Abnormal Returns (CAR) of companies' stock prices following the climate protests. My results indicate that only companies that greenwash the most, experienced relative decreases in their CAR, therefore, suggesting that while extreme degrees of greenwashing may be penalized by investors, it may remain undetected when it is less pronounced.

Finally, I use the same procedure to investigate the effects on the growth of sales of the abovementioned companies. I argue that, if a company engages in greenwashing in their corporate communications, this behaviour will also be reflected in their consumer-oriented marketing. Based on this, I again use the $CTI$ to test whether the sales growth of greenwashing companies has been substantially impacted following the shock. Results indicate that, on average, greenwashing more does not impact the sales growth of a company, for all but the highest greenwashing quartile. For companies in that group, greenwashing has a positive effect on the outcome variable, suggesting that consumers act differently from investors, as high greenwashing intensity has a positive effect on the sales growth. This result suggests that consumers may be convinced by environmental communication, which aligns with the findings of \textcite{schmuckMisleadingConsumersGreen2018,parguelCanEvokingNature2015}. Moreover, the dynamic results show that firms engaging in higher greenwashing have experienced a small, and temporary increase in their sales growth following the shock. 

% ---------------- add to the lit review and think how you described results --------------------
% The primary substantive contribution of this study is to use novel data and methods to quantify the importance of information in consumer choice in an important real-world market.
This study is related to two streams of literature.\footnote{I think I should expand on this and describe the studies a bit more (one line about what they do). Also consider whether a third contribution is needed!} First, it contributes to the body of work concerning the link between financial markets and company's environmental performance. Within this literature, some studies focus on the causes of greenwashing \parencite{servaesImpactCorporateSocial2013,berroneDoesGreenwashingPay2017,kimGreenwashVsBrownwash2015,testaDoesItPay2018}, with a number of researchers arguing that it can be explained with signalling theory \parencite{lyonGreenwashCorporateEnvironmental2011,wuBadGreenwashingGood2020}. Other studies focus on the reaction of the market to stakeholder pressure caused by climate-related events \parencite{birindelliJustBlahBlah2023,schusterStockPriceReactions2023, diaz-raineyTrumpVsParis2021, bouzzineStockPriceReactions2021,cartellierCanInvestorsCurb2023, }. A third set of findings concerns the effects of greenwashing on the financial performance of a company \textit{after} company's greenwashing is exposed \parencite{tetiDoesGreenwashingAffect2024, karpoffReputationalPenaltiesEnvironmental2005, konarDoesMarketValue2001,torelliGreenwashingEnvironmentalCommunication2020}. This paper complements prior literature, by investigating whether it is profitable for a company to greenwash in presence of rising stakeholder pressures, and \textit{before} its deceptive actions are uncovered.\footnote{Look at remarks by David.}


Second, it contributes to literature on the link between consumer behaviour and deception. A number of studies focusing on this topic find that deceptive information, may cause consumers to make less informed choices \parencite{raoDemandHealthyProducts2017,bronnenbergPharmacistsBuyBayer2015}. For example, \textcite{barrageAdvertisingEnvironmentalStewardship2020}, finds that BP's environmental marketing campaign dampened the negative reactions of consumers following an oil spill in which the company was involved. Others find that consumers can be influenced by environmental messaging \parencite{parguelCanEvokingNature2015,schmuckMisleadingConsumersGreen2018,nyilasyPerceivedGreenwashingInteractive2014}. This study complements this literature by testing whether companies can reap the benefits of deception in presence of an increase in consumer activism.

The remainder of the paper is organized as follows. Section~\ref{sect:background} describes the events leading up to and after the shock and presents the argument for the choice of the treatment variable. Section~\ref{sect:data} outlines the procedures for collecting and processing the data. Section~\ref{sect:methodology} summarizes the used methodology and compares it with the typical difference-in-differences framework. Section~\ref{sect:results} describes and interprets the results. Section~\ref{sect:discussion} describes the limitations and concludes.

\section{Background}\label{sect:background}

\subsection{Timeline} 

The year 2019 was an important year for climate change awareness as well as for climate change policy. The Fridays For Future (FFF) movement started in August 2018 by Greta Thunberg gained in popularity following the release of a previous IPCC report on the commitments needed to stop the rising world temperatures. At the same time, 2019 witnessed many climate-related disasters. It was the second-hottest year since the start of the record keeping in 1880 \parencite{noaa2019global}. According to The Guardian, extreme events such as floods, droughts, storms and wildfires have accounted for more than \$100bn worth of damages \parencite{harvey2019climate}. This effect is also visible in the news coverage of that year. As evidenced by the Figure~\ref{fig:news_trends}, 2019 marked a significant increase in new articles related to climate change, environment, and sustainability compared to the previous years. While the overall number of new articles created by the New York Times has been in decline from 2012 until 2019, authors have been writing more articles revolving around these 3 topics.


% \vfill
\begin{figure}[t]
    \caption{Growth in New York Times Coverage by Topic}\label{fig:news_trends}
    \includegraphics[width=\textwidth]{nyt_articles.pdf}
    \captionsetup{font=footnotesize}
    \caption*{Number of new articles published per year (2010--2023). Red line indicates the year of the shock. Source: New York Times}
\end{figure}

\begin{figure}[t]
    \caption{Google Search Trends of ``Fridays For Future''}\label{fig:fff_trend}
    \includegraphics[width=\textwidth]{fff_trend.pdf}
    \captionsetup{font=footnotesize}
    \caption*{Proportion of searches relative to the most searched, for the whole world. Red dots show major strikes organized my FFF.\@ These strikes happened on the following dates (from left to right): 15--03--2019, 24--05--2019, 20--09--2019, 27--09--2019, 29--11--2019. Source: Google Trends}
\end{figure}

The pressure exerted on policymakers culminated in September during a series of strikes titled ``Global Week for Future'' (23--27 September 2019), during which the United Nations Climate Action Summit (UNCAS) took place. In the prior and during that week, FFF organized two global strikes across +270 countries that were attended by 4 million (for both the 20--09--2019 and 27--09--2019 ones) according to the organizers \parencite{fffStrikeStats}. The number of participants as well as the interest were an order of magnitude larger than the previously organized events. Moreover, it was the first time when the youth movement was joined by adults with \citeauthor{watts2019climate} of The Guardian describing it as:

\begin{quote}
    \small
    \textit{For the first time since the school strikes for climate began last year, young people called on adults to join them – and they were heard. Trade unions representing hundreds of millions of people around the world mobilised in support, employees left their workplaces, doctors and nurses marched and workers at firms like Amazon, Google and Facebook walked out to join the climate strikes.} \parencite{watts2019climate}
\end{quote}

As shown on the Figure~\ref{fig:fff_trend}, strikes related to that week attracted over twice as much attention online as any other strikes organized by FFF.\footnote{Figure~\ref{fig:cc_trend} shows a similar trend for a ``Climate Change'' query.} This effect has been amplified by speeches given by Greta Thunberg at the climate strike in New York City and at the UNCAS in which she motivated the strikers as well as attacked the governmental officials for not taking appropriate action on climate change \parencite{2019thunberg}. 

These events had an impact on financial markets. In a \citeyear{schusterStockPriceReactions2023} study,~\citeauthor{schusterStockPriceReactions2023} found that the FFF global climate strikes significantly affected the short-term stock performance of firms from the S\&P 500 and STOXX Europe 600 indices. They estimated a financial event study and find that the two global climate strikes in September 2019 generally hurt the performance of the companies in these indices. Moreover, their findings suggest, that based on ESG scores, firms with lower environmental performance suffered more those that perform better in this metric.  This is in line with the findings of a theoretical model by \textcite{pastorSustainableInvestingEquilibrium2021}, which argues that sustainable assets outperform other assets, when concerns for sustainability rise.

Overall, the action taken by the FFF grasped the imagination of the public with many people choosing to walk out onto the streets and support their causes. The events gathered news and online attention and likely had an impact on the performance of companies in that period.


\subsection{ESG Reports}


This study uses broadly defined responsibility reports for the construction of the main measure of greenwashing. These reports contain information meant for investors and stakeholders. They inform the reader on initiatives undertaken by the company in various areas related to corporate social responsibility (CSR). This study focuses on the sustainability section of these reports, that contains information on the impact that the company has on the environment, its production standards, as well as the climate risk a company is exposed to. To illustrate, \citeauthor{walmart2018csr}'s 2018 Global Responsibility Report contains a dedicated sustainability section that describes the impact of Walmart's activities on the environment. This report also contains a dedicated ``ESG commitments'' section, that outlines the company's ESG goals and their progress on attaining them.\footnote{This Walmart Inc.~report, as well as others can be easily accessed \href{https://www.responsibilityreports.com/Company/walmart-inc}{\underbar{here}}.}

These reports are desirable for a number of reasons. Firstly, literature points to the fact that investors care about sustainability and climate risks, and that they reward companies when these report on actions related to them \parencite{ilhanClimateRiskDisclosure2023,kruegerImportanceClimateRisks2020,pastorSustainableInvestingEquilibrium2021,testaDoesItPay2018}. From the firm perspective, corporate communication allows it to raise its market value, when they inform their investors sufficiently \parencite{servaesImpactCorporateSocial2013}. Furthermore, by communicating on their ESG activity, they can increase their legitimacy to its stakeholders \parencite{torelliGreenwashingEnvironmentalCommunication2020}. From this one can conclude that it is in the interest of these companies, to communicate as much information as possible through this channel, to reap the rewards. 

However, as these reports do not fall under the oversight of any institutions (unlike e.g.,~annual earnings reports), they allow companies to introduce deception in their communications. For example, the companies issuing these reports, may overstate their environmental achievements, or understate the damages that they caused, and as such use them to appeal to their stakeholders. There can be many reasons why this behaviour can be incentivized. To illustrate, if investment in sustainability is unobservable, companies may benefit from overstating their actual investments \parencite{wuBadGreenwashingGood2020}. Alternatively, if their activity negatively affects the environment, they may desire to diminish the negative reputational impacts of their actions \parencite{marquisScrutinyNormsSelective2016, binglerCheapTalkCherrypicking2022}.\footnote{For a more comprehensive review, see~\cite{kimGreenwashVsBrownwash2015}} In addition to over-/under-reporting, companies can also utilize other strategies to impact the perception of the stakeholders. For example, \textcite{parguelCanEvokingNature2015,schmuckMisleadingConsumersGreen2018} find that referring to nature and including nature-related graphics in their communications can positively influence their perception of the company. Given all of these reasons, they seem to provide a rich source of (possible) greenwashing behaviour.


Finally, they are a good measure of greenwashing, as they come directly from the company and thus are free from interference of other parties. For example, some researchers rely on the use of ESG scores in their analyses.\footnote{Cite something here (maybe from studies cited below).} However, as these scores are calculated by external providers, they can be biased, which can be seen given inconsistencies of scores, between different rating agencies \parencite{bergAggregateConfusionDivergence2022, chatterjiRatingsFirmsConverge2016}. This suggests that they may not provide a good estimate of company's environmental performance. Therefore, I refrain from using them as my measure of greenwashing. Instead, I chose to use the underlying reports, and proxy greenwashing following the method of~\cite{binglerHowCheapTalk2024}. The method, together with the data collection procedure, are described in the next section.\footnote{Probably also re-write a bit.}


\section{Data}\label{sect:data}

\subsection{Cheap-Talk Index}

The ESG reports were scraped from the \href{https://responsibilityreports.com}{ResponsibilityReports.com}. Additionally, the website allowed me to retrieve information on company headquarters location and the approximate number of employees.\footnote{Table~\ref{tab:sust_reps} in Appendix~\ref{app:extra_figs} displays the number of reports available per year. The number of published reports increases with time, as communicating sustainability has become more important.} As the analysis concerns the events of 2019, the scraped data was subset for companies reporting in 2018, to guarantee that the reports did not originate in response to the upcoming shock. This resulted in 852 reports ready for analysis. 

To identify deception in relation to the environmental claims of companies, I implemented the method outlined in \textcite{binglerHowCheapTalk2024}. In this study, the authors create the climate Cheap-Talk Index ($CTI$), based on the content of annual financial reports. They use the extracted text to construct and train machine learning models that are then used to examine the level of cheap-talk in a given report. 

Their approach is motivated by the work of \textcite{coenAreCorporateClimate2022, ramusWhenAreCorporate2005}. While the ESG reports can allow for greenwashing (as outlined in the previous section), it is hard to measure it. However, the authors of these two studies find a way. They argue that companies are less likely to follow-up on their commitments if these are stated in a vague and non-specific manner. Therefore, the $CTI$, interpreted as a measure of non-specificity of companies climate commitments, can be used as a proxy for greenwashing, under the assumption that the vagueness reflects the deviation of claims of companies, from their actions. This resembles the argument made in \textcite{binglerCheapTalkCherrypicking2022,marquisScrutinyNormsSelective2016}, which argue that companies may want to under-report their impact on the environment if it is negative. However, it must be noted that this is only an approximation, that can suffer from measurement error. Nonetheless, under these conditions it allows for the estimation of the perceived greenwashing of companies. 

Their method consist of two steps. First, they access an instance of the ClimateBert model developed by \textcite{webersinkeCLIMATEBERTPretrainedLanguage2022}. ClimateBert is a transformer-based language model, that was pre-trained on climate-related texts, in addition to regular training. This process, allows the ClimateBert to be more precise in various tasks (such as text classification), because through this pre-training, it became more exposed to domain-specific vocabulary of climate related information. According to the authors, this resulted in a decrease in error rates of the model of up to 35.71\% depending on the task.~\citeauthor{binglerHowCheapTalk2024} fine-tune this instance of the model for classifying paragraphs as \textit{climate-related}. This means, that they applied further training to the model, to make it better at predicting whether the text extracted from annual reports is climate related (e.g.~mentions sustainability initiatives) or not. 

As a second step, they train another instance of the same model to classify paragraphs as \textit{specific} (containing specific commitments) and \textit{non-specific} (containing only vague commitments). This means that they additionally post-train the model, so that it can better differentiate between paragraphs that refer to specific goals of the company (such as ``Achieve an 18 percent emissions' reduction in Walmart's own operations by 2025 (over 2015 baseline)'' \parencite{walmart2018csr}) and paragraphs that only contain vague statements about climate-related issues (such as ``Walmart believes business can play a role in addressing climate change by reducing GHG emissions, and our investments in renewable energy and efficiency underscore that belief''). The procedures that the authors used for training of both of the models are described in \textcite{binglerCheapTalkCherrypicking2022,binglerHowCheapTalk2024}. Then, they use the set of climate-related paragraphs and the set of specific paragraphs within a report to create the $CTI$. Equation~\ref{eq:cti} presents the formula for the index:
\begin{equation}\label{eq:cti}
    CTI_i = \frac{CLIM_i \cap NONSPEC_i}{CLIM_{i}},
\end{equation}
where for a report of company $i$, $CLIM_i$ represents the number of climate-related paragraphs, $NONSPEC_i$ represents the number of paragraphs containing non-specific commitments, and $CLIM_i \cap NONSPEC_i$ represents the intersection between the two. 


In order to apply this method to my data, I proceeded as follows. First, I extracted the text from the reports. This has been done by using blocks contained in the PDF files to preserve the structure of the paragraphs.\footnote{By \textit{blocks}, I mean an object within the structure of a PDF file that usually determines the location of a paragraph on a page.} To further refine the data, each block has been classified as a paragraph, if it spanned at least 3 lines, contained at least 20 words and had at least one full stop. This pre-processing procedure is the same as in \textcite{binglerHowCheapTalk2024}. Then, I applied the models trained by the researchers to establish which paragraphs are climate related, and which contain specific commitments.\footnote{Both of these models can be downloaded from \href{https://huggingface.co/climatebert}{\underbar{here}}} Finally, I aggregated the $CTI$ into quartiles, to create the final treatment-dose variable $D_i$. The aggregation was carried out, because the index is very sensitive to the number of climate related paragraphs, as they enter in the denominator. This implies that small differences in the number of climate-related  paragraphs can disproportionally impact the score of the company. Therefore, aggregating the scores into quartile-bins reduces noise.\footnote{If a report of a hypothetical company contains 5 climate related paragraphs, and 4 out of them are non-specific, $CTI_i=0.8$. However, if the number of classified paragraphs rises to 6, the $CTI$ drops by 14 p.p.}



\subsection{Outcome Variables}

The Refinitiv dataset was used to retrieve data on firms' characteristics, their price indices and their quarterly sales. Moreover, it was also used to retrieve the prices for regional market indices (e.g.\ S\&P500, MSCI Europe, etc.) that were needed to estimate the normal returns for each company. The characteristic variables include the stock exchange of the listing, country of incorporation, country of domicile, the company's sector and the company's industry. The stock price data and sales data required distinct pre-processing methods. The following section first details the procedure for the stock data, followed by that for the sales data.

First, I log-differenced the price indices (defined as the stock price expressed as a fraction of the initial listing price) to transform them into percentage changes. The same has been done for the values of regional stock indices. I chose the daily frequency for the stock price data, which was motivated by the likely short-term effect on stock prices, as found by \textcite{schusterStockPriceReactions2023}.\footnote{Table~\ref{tab:data_freq} in Appendix~\ref{app:extra_figs} shows the total number of periods that were retrieved.}

% \subsubsection{Abnormal Returns}

In order to plausibly estimate the effect of the treatment on stock returns, I needed to separate the effects of the market movement from the movement of the stock of the company itself. These abnormal returns were estimated following the market model approach outlined in~\textcite{mackinlayEventStudiesEconomics1997}. This approach relies on estimating the expected stock return by predicting it using the return of the whole market. As the sample contains companies from multiple continents with potentially differing markets, each company was allocated to a geographical market index by their country of domicile.\footnote{The allocation can be viewed in Table~\ref{tab:mrkt_ind} in the Appendix~\ref{app:extra_figs}.} If the country of domicile was missing, the regional index was based instead on the country of listing. The model was separately estimated for each company using 212 daily observations that predate the estimation window of interest. Below I report the regression specification used for the estimation:
\begin{equation}\label{eq:reg_market_model}
    RN_{it} = \alpha_i + \beta_{i} RI_{mt} + \varepsilon_{it},
\end{equation}
where $RN_{it}$ is the return of the security $i$ at time $t$ predicted by the return of its geographical market $m$, $\alpha_i$ is the regression constant of for the company $i$, and $RI_{mt}$ is the return of the said market. As this regression estimates the return of the security predicted by the return of the market, the abnormal return for $i$ in a given time period $t$ is given by the residual $\varepsilon_{it}$.

This type of modelling is sufficient for my use case, as per the work of~\textcite{mackinlayEventStudiesEconomics1997}, employing more sophisticated models (e.g.~factor models) yields limited improvement in studies that do not rely on within-group variation, such as within-sector variation.

After estimating the model, I used it to predict the abnormal returns within the event window. Then, the abnormal returns for each of the securities were cumulatively summed to arrive at the Cumulative Abnormal Return (CAR). This was done, as aggregation across time is needed in order to draw useful inference from the results of the actual study. This results from the fact that estimating the average abnormal returns directly would not take into account the path of the data following treatment, but only the instantaneous effect of the treatment dose.
% \footnote{Make clearer with respect to the reasoning and my later DiD}

For the sales data, the procedure only involved log-differencing of the data to arrive at percentage changes. However, as changes in sales can be seasonal, I instead create Year-on-Year growth rates, by using the difference of sales relative to the same quarter last year. The choice of quarterly frequency was motivated by the data availability, as it was the lowest frequency available. 


\subsection{Merging}

In the last step of the data preparation, I merged the dataset containing $CTI$ scores of companies, with the dataset containing the stock prices, and the dataset containing quarterly sales of companies. This process was difficult, because company names are often reported differently between different services. In the process of merging, 71 observations were dropped for the stock price dataset, because they could not be merged. The final dataset contained 781 companies. For the dataset containing quarterly sales, the loss in values was more severe, because of data availability. Unfortunately, the Refinitiv database did not contain the quarterly sales for all companies in the sample, which further the number of companies to 503. Tables with descriptive statistics for each outcome variable can be viewed in the Appendix~\ref{app:extra_figs}.


\section{Methodology}\label{sect:methodology}

This section describes the methodology used in the attempt to estimate the causal effect of greenwashing on company performance.
My initial approach was to use a classic Difference-in-Differences (DiD) methodology with binary treatment, however upon consideration I decided to implement a variation of DiD with multivalued treatment following \textcite{callawayDifferenceinDifferencesContinuousTreatment2025}. This approach is motivated by the fact, that the market may react differently to firms that use more cheap talk in their communication. This method also allows for estimation of experiments in settings without an untreated group, where the group with the lowest treatment-intensity is used as a comparison. 

Given the use of newly developed methods in this study, I chose to contrast their methodology with the well-understood traditional DiD framework to better illustrate how this approach can improve upon the default approach. Therefore, the methodology is divided into three parts. Firstly, I will cover the conventional difference-in-differences design. Then I will contrast it with the implemented design using multivalued treatment. Finally, I will describe the differences in the taken approach when estimating the effect on sales of companies instead of stock prices.


\subsection{Default Difference-in-Differences}

Conventional Difference-in-Differences studies aim to estimate the causal effect of a policy by comparing units over time and then comparing those changes across groups. In order to separate the causal effect, they require two main assumptions explained below.\footnote{A formal treatment of these assumptions can be found in Appendix~\ref{app:models}} 

\textbf{Parallel trends (PT) assumption}, which requires that in the absence of treatment, the outcomes of the units assigned to the treatment group would have evolved in the same way (in terms of the trend) as the outcomes of the units that were not assigned to treatment. In my example, it implies that in the absence of cheap-talk of a company, their stock price/sales would have evolved in the same way as the stock price/sales of those that do not engage in cheap talk. 

\textbf{No-anticipation assumption}, imposes that the individuals do not act in anticipation of the treatment. In my example, it implies that before the shock, the investors do not expect the greenwashing firms to perform differently from non-greenwashing firms after the shock, and therefore choose not to adjust their portfolios in advance of it. In other words, belonging to the treatment or the control group should not matter for the outcomes of the companies before the climate strikes shock.

Under these two assumptions and an additional assumption on independent sampling, the effect of obtaining treatment can be recovered using the following regression specification utilizing two-way-fixed-effects (TWFE):
\begin{equation}
    Y_{i,t} = \alpha_i + \Phi_t + \beta^{\text{twfe}} D_{i} \cdot Post_{t} + \varepsilon_{i,t},
\end{equation}
where $Y_{it}$ is the outcome variable, $\alpha_i$ is the entity fixed effect, $\Phi_t$ is the time fixed effect, $D_i$ is the binary treatment dummy, $Post_t$ is the post-treatment period dummy. In this setting, $\beta^{\text{twfe}}$ recovers the average treatment effect.

Although the interpretation of this treatment effect has been well described, it is not useful when analysing cases with multiple treatment doses, as it can be shown that the TWFE parameter becomes a weighted average of dose-$ATT$'s, with weights that deprive this parameter of its causal interpretation \parencite{callawayDifferenceinDifferencesContinuousTreatment2025}. Therefore, in the following section, I describe the methodology of an adjusted framework.

\subsection{Multivalued Difference-in-Differences}

In order to incorporate the multivalued treatment into my study, I implemented the framework developed by~\textcite{callawayDifferenceinDifferencesContinuousTreatment2025}. This is not the only approach present in the literature (see~\cite{dechaisemartinDifferenceinDifferenceEstimatorsContinuous2024}), however it is the most suitable given the setting of the study. This method was developed in the context of estimating the average treatment effect in settings where there is no treatment in pre-event periods.\@ \citeauthor{dechaisemartinDifferenceinDifferenceEstimatorsContinuous2024} differs in this regard, as it relies on treatment being positive in the periods prior to the event, and it requires a change in intensity in the event-period, to retrieve the average treatment effect. As in my case the treatment is fixed for an entity, but varies in intensity between them, this requires the~\textcite{callawayDifferenceinDifferencesContinuousTreatment2025} approach. It proposes the use of a multivalued treatment, where the single dummy $D_i$ can be replaced with multiple dummies $D_{i,d} \in \mathcal{D}$ that show that the unit $i$ has received the dose $d$, where $d$ refers to the ``bins'' that signify different treatment intensity. Then, this approach compares the dose of a group $D_{i,d}$, to the untreated group $D_{i,0}$.

This framework shares the no-anticipation and the independent sampling assumptions with the conventional DiD literature, however the differences arise in the parallel-trends assumption. In this setting, the traditional PT assumption allows for the estimation of each of the individual intensity parameters (i.e. $ATT$ for each treatment is still recovered), however, a stronger assumption is needed to compare the effects across these estimators. Therefore, this assumption is required in order to assess the effect of higher greenwashing intensity of a firm.  

The \textbf{Strong Parallel Trends (SPD) assumption} requires that the average evolution of  outcomes for the entire treated population if all experienced dose $d$ is equal to the path of outcomes that dose group $d$ actually experienced \parencite{callawayDifferenceinDifferencesContinuousTreatment2025}. This assumption is called the strong parallel trends, as it implies that the average response to the treatment would be the same for all dose groups, if they were to receive (for example) a lower dose $d$. It is more restrictive than the regular parallel trends assumption, and the researchers themselves argue that it may be unreasonable in many cases due to treatment effect heterogeneity of different groups. However, under this assumption, one can estimate the average causal response parameter, which the researchers denote as $ACRT$. They show that the $ACRT$ is equal to the difference in average potential outcomes between dose level $d$ and the next lowest dose, scaled by the difference between these two doses. 

Furthermore, the authors extend this framework to allow for estimation of treatment effects, where there are no untreated units. Instead, all dose-groups are compared to the lowest-dose group. In this case, the $ATT$, which typically shows the effect of switching between no treatment to a treatment dose $d$, cannot be recovered. Nonetheless, they show that one can instead estimate the change in outcomes resulting from switching from low-intensity treatment to a high intensity treatment. While this comparison would introduce a selection bias under standard parallel trends, they show that one can avoid it by imposing strong parallel trends. Since my setting lacks explicitly untreated companies, and instead uses the companies in the lowest quartile of the CTI index as the comparison group, this assumption is required for the whole estimation process. In order to retrieve the causal effect, I estimate the following multivalued specification:
\begin{equation}
    Y_{i,t} = \alpha_i + \phi_t + \sum_{d=2}^{4} \beta_d \cdot 1\{D_{i,d}=d\}\cdot Post_t + \varepsilon_{i,t},
\end{equation}
where $Y_{i,t}$ is the outcome variable, $\alpha_i$ are the unit fixed effects, $\phi_t$ are the time fixed effects, $1\{D_{i}=d\}$ is a dummy variable equal to $1$ if the unit $i$ belongs to the treatment group $d$, $Post_t$ is the dummy variable equal to one in the period after the FFF climate strike of September 20th, 2019, and $\varepsilon_{i,t}$ is the residual. The standard errors are clustered at the company ($i$) level, following \textcite{abadieWhenShouldYou2022}. Note that $d=1$ is omitted, as the regression is estimated relative to the lowest dose group. In this specification, each $\beta_d$ estimates the difference between $ATT(d)$ and the $ATT(d_L)$, which is average difference in outcomes when treated with $d$ instead of $d_L$. Formally, it can be expressed as: 
\begin{equation}
    \beta_d = E[Y|D = d] - E[Y |D = d_L] = ATT(d) - ATT(d_L) = E[Y_{t=2}(d) - Y_{t=2}(d_L)]
\end{equation}
This is a common approach adopted by many applied researchers \parencite{acemoglu_finkelstein_medicare,deschenes_greenstone_clim_change}. While this approach does not summarize the effect of the policy with one parameter, in the same way as the $\beta^{\text{twfe}}$ does, \citeauthor{callawayDifferenceinDifferencesContinuousTreatment2025} argue for this as their preferred specification. According to their study, estimating these parameters separately omits the weighting problem faced by specifications such as:
\begin{equation}
    Y_{i,t} = \alpha_i + \phi_t +  \beta^{\text{twfe}} Post_t \cdot (d_j - d_{j-1}) + \varepsilon_{i,t},
\end{equation}
where the $\beta^{\text{twfe}}$ tries to aggregate the dose-specific $ATT$'s into a single number. They propose their own aggregation scheme that estimates the local average treatment effect ($ATT^{loc}$), and the local average causal response ($ACRT^{loc}$) which operate by weighing the treatment doses by the probability of observation $i$ being in dose group $d$. However, this aggregation is only described within the context of having untreated observations. As in this study the level treatment effects are estimated relative to a group treated at a lower intensity instead, I refrain from summarizing the results in this way, and present dose specific results in the next section.


\subsection{Validity of Assumptions --- Stock Prices}

The chosen framework requires the strong parallel trends and the no-anticipation assumptions. With respect to the SPT, this assumption may be concerning, as in the same way as in \textcite{koenigImpulsePurchasesGun2023}, the companies self-select into being treated. To alleviate this concern, Figure~\ref{fig:stock_trend} shows that the firms stock prices followed similar paths no matter their treatment assignment.
% \footnote{Expand more your description of the picture} 
Furthermore, an event study is run in the next section to evaluate the possibility of pre-trends. However, it has to be noted that the authors of \textcite{callawayDifferenceinDifferencesContinuousTreatment2025} state that this method of evaluating the existence of pre-trends has low power, and therefore the lack of significance within the pre-trends does not necessarily imply that they actually do not exist. Nonetheless, the event study is estimated as per the standard practice.
With respect to the no-anticipation assumption, concerns are more pronounced. In their event study, \textcite{schusterStockPriceReactions2023} found evidence of anticipation of the stock market within a period of 20 days prior to the FFF strikes. In order to alleviate this concern, an analysis is performed with the shock date being shifted to multiple different dates preceding the shock.

\begin{figure}[t]
    \caption{Mean Cumulative Abnormal Return (Daily)}\label{fig:stock_trend}
    \centering
    \includegraphics[width=0.9\textwidth]{stock_d_trend.pdf}
    \captionsetup{font=footnotesize}
    \caption*{Treatment Groups indicated with numbers from (lowest dose) 1 to 4 (highest dose). Red line indicates the first Fridays For Future climate strike during the Global Week For Future. Source: Refinitiv}
\end{figure}

\subsection{Validity of Assumptions --- Sales}

The analysis of greenwashing effects on quarterly sales requires the previously stated strong parallel trends and no-anticipation assumptions, but two additional concerns emerge.

With respect to the strong parallel trends assumption, there are concerns that the firms' outcomes could perform differently due to self-selection. From Figure~\ref{fig:sales_trend}, it appears that although the trends of all treatment groups move in a similar direction, the outcomes of firms in the third quartile of the $CTI$ seem to diverge before the quarter of the shock. This could suggest a violation of SPT, however this question is also evaluated with an event study.

The non-anticipation assumption could be violated in certain cases. For example, if the consumers were to stop buying goods from brands that they perceive as greenwashing before the start of the protest, this could violate the assumption. One theory that could support this argument is the work by \textcite{liaukonyteFrontiersSpillingBeans2023}. In this study, the authors find that \textit{online} boycotting activity significantly impacted the sales of the boycotted companies. As the large amount of information related to FFF climate strikes spread on social media, this could violate the assumption in question, if the boycotts happened in quarters preceding the quarter of the shock. However, in preparation of this study, I found no calls for boycotting activities in advance of the quarter which includes the climate strikes. 


\begin{figure}[t]
    \caption{Year-over-Year Change in Quarterly Sales}\label{fig:sales_trend}
    \centering
    \includegraphics[width=0.9\textwidth]{sales_trend.pdf}
    \captionsetup{font=footnotesize}
    \caption*{Treatment Groups indicated with numbers from (lowest dose) 1 to 4 (highest dose). Red line indicates the quarter which witnessed the Global Week For Future and the FFF climate strikes. Source: Refinitiv}
\end{figure}

Furthermore, it has to be noted that this analysis may be significantly impacted by the loss of heterogeneity due to the sample size loss.  

Finally, one must additionally acknowledge one more assumption with respect to this part of the study. In this setting, the $CTI$ works only as a proxy to the belief of consumers about the greenwashing-intensity of the companies. This is not an obvious assumption, as firms may try to be more (or less) deceitful to consumers than to investors. As a result, there is potential for  measurement error in the estimated coefficients. For example, a company may greenwash in their ESG reports pointed at investors, but may focus more on other aspects of their business in their marketing. In that case, the firm's greenwashing directed at the consumers would be less than their $CTI$ implies, which would bias the coefficients of this study towards zero. On the other hand, if the firm would be more misleading towards consumers, this could imply a positive bias, as the firms' communications could be more focused on the environmentally-friendly information than implied by the $CTI$. Nonetheless, for the bias affect the estimates, it would require that firms systematically act in this way. Given this fact, and the lack of studies on the relation between marketing, ESG reporting and ESG performance, it seems plausible to assume that while the error is possible, it would be small due to the size and heterogeneity of the sample. Accordingly, this study takes as given that firms will, on average, communicate similarly with both consumers and investors, and therefore, that their ESG reports will reflect their sustainability-oriented marketing. 

% \newpage

\section{Results}\label{sect:results}


\subsection{The Response of Stock Prices to Greenwashing}\label{subsect:stocks}

Column (1) in Table~\ref{tab:main_results} presents the results of estimation of the model on the cumulative abnormal return of companies. The model was estimated on a 40-day event window, with 20 days preceding the shock and 20 days following the shock. The length of the window was based on the expected short-term effects of the shock, following \textcite{schusterStockPriceReactions2023}. Moreover, it was purposefully made short, to make sure that the effects are not contaminated with behaviour related to the COP25 conference that happened at the beginning of December 2019.\footnote{Note that due to the stock market not working on weekends, the actual number of estimated periods equals 28.} This process resulted in three coefficients of interest --- $\beta_2$, $\beta_3$, and $\beta_4$. These coefficients show the response of the stock market to the climate activism of the Global Week For Future, differentiated by the greenwashing-intensity of companies. In this specification, all three coefficients are negative and insignificant. This implies that statistically, higher greenwashing intensity proxied by the $CTI$ index, did not impact the behaviour of the stock prices over time, relative to the sample of firms that greenwash the least. 


\begin{table}[t]
    
    \captionof{table}{Results of Estimation}\label{tab:main_results}
    
    \centering
    \begin{tabular}{l c c c c }
        \toprule
        \toprule
        & \multicolumn{3}{c}{CAR}     & \multicolumn{1}{c}{$\Delta_4$ Log of Sales} \\
        Variables                   & {(1)}           & {(2)} & {(3)}   & {(4)}\\
        \midrule
        $Post \times D_2$           & -0.002       &  -0.001   & -0.002  & 0.027\\
                                    & (0.005)      &  (0.005)  & (0.006)  & (0.018)\\
        $Post \times D_3$           & -0.002       &  -0.000   & 0.004  & 0.023\\
                                    & (0.005)      &  (0.005)  & (0.006)  & (0.018)\\
        $Post \times D_4$           & -0.008       &  -0.006   & -0.004  & 0.040$^{**}$\\
                                    & (0.005)      &  (0.005)  & (0.006)  & (0.020)\\
        Entity FE                   & {$\checkmark$}  & {$\checkmark$}  & {$\checkmark$}  & {$\checkmark$}\\
        Time FE                     & {$\checkmark$}  & {$\checkmark$}  & {$\checkmark$}  & {$\checkmark$}\\
        Firms                       & {781}           & {781}  & {781}  & {503}\\
        Time Periods                & {28}            & {28}  & {28}  & {7}\\
        Observations                & {21,868}        & {21,868}  & {21,868}  & {3,521}\\
        R$^2$                       & {0.90065}       & {0.90055}  & {0.90057}  & {0.39530}\\  
        \bottomrule
    \end{tabular}
    
    \vspace{0.2cm}

    \begin{tablenotes}
        \footnotesize
        \item The sample period for columns (1), (2), and (3) lasts from September 1st, 2019 until the 10th of October 2019. The data for (4) covers the period from Q3 2018, until the Q1 2020. Column (1) shows the treatment for the shock on the 20th of September 2019. Columns (2) and (3) show the result when the treatment is moved taken as five and ten days earlier (respectively). Column (4) uses the Q3 2019 as the treatment period. Time FE is at the day level for columns (1), (2) and (3) and quarter level for column (4). Standard errors are clustered at the firm level and are in parentheses. $^{*}p<0.1, \text{ } ^{**}p<0.05,\text{ } ^{***}p<0.01$.
    \end{tablenotes}
\end{table}


Table~\ref{tab:acrt} displays the dose-wise average causal responses for each dose group. As described in Section~\ref{sect:methodology}, this parameter can be interpreted as the effect of increasing a company's greenwashing intensity from dose $d_{i-1}$ to dose $d_i$. As can be seen from column~(1), both the $ACRT$ of switching from $d_2$ to $d_3$ and the  $ACRT$ of switching from $d_3$ to $d_4$ are insignificant. However, one can notice that the second effect is negative. 

\begin{table}[t] \centering
    \captionof{table}{Average Causal Responses}\label{tab:acrt}
    
    \begin{tabular}{l S[table-format=3.2] S[table-format=3.2]}
        \toprule
        \toprule
                            & {CAR}           & {$\Delta_4$ Log of Sales} \\
        Variables           & {(1)}           & {(2)}\\
        \midrule
        $ACRT(D_3)$         & 0.000        & -0.004 \\
                            & (0.005)      & (0.020)\\
        $ACRT(D_4)$         & -0.007       & 0.017\\
                            & (0.015)      & (0.021)\\
        \bottomrule
    \end{tabular}
    
    \vspace{0.2cm}

    \begin{tablenotes}
        \footnotesize
        \item Average causal response estimator by \textcite{callawayDifferenceinDifferencesContinuousTreatment2025}. $ACRT(D_i)$ describes the effect of changing the treatment dose from $d_{i-1}$ to $d_i$. $^{*}p<0.1, \text{ } ^{**}p<0.05,\text{ } ^{***}p<0.01$
    \end{tablenotes}
    
\end{table}


In addition to the DiD estimation, I also estimated an event study using the specification from column~(1), to assess the dynamic effects of greenwashing. These results are plotted in Figure~\ref{fig:eve_stock}. Each panel displays the dynamic treatment effects for a single treatment-dose group (relative to $d_1$). While the average effect estimated in column~(1) of Table~\ref{tab:main_results} for the highest treatment-dose group is insignificant, the dynamic treatment effects are significant and negative in the post-treatment period. This suggests that having the highest greenwashing-intensity may have a negative impact on the cumulative abnormal returns of a firm, relative to the cumulative abnormal return if the firm engaged in the lowest greenwashing intensity. However, this effect is small and temporary. Moreover, this plot can be used to analyse possible violation of the strong parallel trends assumption. In the panel (a) of the figure, one can see that the pre-trend values for September 16th and 17th are negative. This creates a concern that the assumption may in fact not hold. It is also concerning, given that this test for pre-trends has low power \parencite{rothWhatsTrendingDifferenceindifferences2023}.
                            

Finally, columns (2) and (3) in Table~\ref{tab:main_results} display the performed robustness checks on the no-anticipation assumption. They were estimated with the treatment period being moved up by 5 and 10 days, respectively. As can be seen from the table, the results of this estimation are also insignificant and negative. This suggests that the stock market did not expect the FFF strikes to affect the firms differently with respect to their doses in advance of the strikes. Event studies for these robustness checks can be found in Appendix~\ref{app:robustness}.
% \newpage


\begin{figure}
    \caption{Event Study Results --- Stock Prices}\label{fig:eve_stock}
    \centering
    
    (a) Group 2
    
    \includegraphics[width=0.79\textwidth]{stock_price_eve_2.pdf} \\
    
    (b) Group 3
    
    \includegraphics[width=0.79\textwidth]{stock_price_eve_3.pdf} \\
    
    (c) Group 4
    
    \includegraphics[width=0.79\textwidth]{stock_price_eve_4.pdf}
    
    \captionsetup{font=footnotesize}
    \caption*{The effect of receiving treatment on the cumulative abnormal returns of stock prices. Gaps in the graph are due to the inclusion of weekends in the data. Group number indicates the treatment dose, with 2 being the lowest and 4 being the highest. All estimates are relative to receiving dose $d_1$. Estimation window is set to 20 days before the treatment and to 20 days post treatment to display the short-term effects. 20th of September 2019 marks the first period of treatment and 19th of September 2019 is set as the reference. 95\% confidence intervals are reported.}
\end{figure}

% \newpage


\subsection{The Response of Sales to Greenwashing}

Column (4) in Table~\ref{tab:main_results} presents the results of estimation of the model on the year-on-year quarterly sales of companies. The model was estimated with an asymmetric event window of 4 quarters prior to the treatment and 3 quarters post-treatment. I chose the length of the event window such that the model captures the seasonality of quarterly sales growth in the pre-treatment period, but at the same time such that the effect is not contaminated by other events. The post-treatment period was purposefully chosen to be short, as to avoid the interference with the COVID-19 pandemic. The three estimated coefficients show the response of the quarterly sales growth to the climate activism of the Global Week For Future, differentiated by the greenwashing-intensity of each company. Out of the three treatment doses, it appears that only the highest has a significant coefficient at the 5\% level. Interestingly, contrary to the results of the stock prices, all coefficients are positive. This suggests a possibly different mechanism at play, and implies that having the highest greenwashing-intensity can have a positive effect on the growth of the firm's sales, relative to greenwashing with the lowest intensity.

The $ACRT$'s displayed in column (2) of Table~\ref{tab:acrt} fail to reject the hypothesis, that increasing a firm's dose from $d_2$ to $d_3$, and from $d_3$ to $d_4$ significantly increases its quarterly sales growth. This is a counter-intuitive result, as the model suggests that on average the highest greenwashing level compared to the lowest level significantly affects quarterly sales, however it rejects the hypothesis that the marginal increase in greenwashing from the third dose to the highest dose increases sales. This effect most likely stems from the fact that the standard error of the ACRT is constructed using the variances of both the $ATT(D_3)$ and $ATT(D_4)$ and their covariance. It is the size of these parameters, which likely causes the $ACRT(D_4)$ to stay insignificant.

Finally, in the same way as in the previous subsection, event studies were estimated for each of the treatment-dose groups. As can be seen in Figure~\ref{fig:eve_sales}, for each of the groups, there exist positive and significant dynamic effects in either the first or the second post-treatment period. This would suggest that having a higher greenwashing intensity in fact has a positive but small and temporary impact on sales growth.


\begin{figure}
    \caption{Event Study Results --- Sales}\label{fig:eve_sales}
    \centering
    
    (a) Group 2
    
    \includegraphics[width=0.79\textwidth]{sales_eve_2.pdf} \\
    
    (b) Group 3
    
    \includegraphics[width=0.79\textwidth]{sales_eve_3.pdf} \\
    
    (c) Group 4
    
    \includegraphics[width=0.79\textwidth]{sales_eve_4.pdf}
    
    \captionsetup{font=footnotesize}
    \caption*{The effect of receiving treatment on the 4-quarter-log-difference of quarterly sales. Group number indicates the treatment dose, with 2 being the lowest and 4 being the highest. All estimates are relative to receiving dose $d_1$. Estimation window is set to 4 quarters before the treatment to account for seasonality and to 3 quarters after the treatment to display the effects but exclude the effects from the COVID-19 pandemic. 2019 Q3 is the period of treatment and 2019 Q2 is set as a reference period. 95\% confidence intervals are reported.}
\end{figure}

However, it is advised to interpret these results with caution, as they could be impacted by the significant decrease in sample size, and the COP25 conference that happened in the fourth quarter of 2019. 



\subsection{Interpretation}

Overall, these results suggest different effects of greenwashing depending on the outcome of interest. Due to a concern about the possible measurement errors within the study, I interpret the effects in terms of signs and magnitudes, instead of focusing on the precise values of parameters.\footnote{I think I should add more of an interpretation to this section in the context of the literature.}

The results for cumulative abnormal returns, although insignificant, suggest a negative effect. The average dose-wise treatment effects, imply that greenwashing more than other firms did not have an effect on stock performance in the analysed period. However, the dynamic treatment effects for the highest treatment dose suggest that firms that greenwash the most may be temporarily penalized relative to others, however by a small margin. With respect to the second and third treatment doses, the lack of both average effects and dynamic effects can suggest that there exists some threshold of greenwashing above which firms are penalized, and below which firms' greenwashing is not detected.

The effects for the sales growth of companies are starkly different, and present overall positive results. While the average treatment effects suggest that higher greenwashing does not, on average, affect the growth of sales for all but the highest dose group, the event studies present a more nuanced picture. The significant and positive (dynamic) reaction of sales growth observed across all groups in the post-treatment periods, may suggest that consumers can be misled with environmental information, which would be consistent with previous studies by \citeauthor{schmuckMisleadingConsumersGreen2018} and \citeauthor{parguelCanEvokingNature2015}. Moreover, the large magnitude of the effect for the group receiving dose $d_4$ implies that consumers could positively respond to higher intensity of greenwashing. Nonetheless, all of these dynamic effects are temporary and last only a quarter.

% \newpage
\section{Conclusion}\label{sect:discussion}
% Summarize again the context, the findings and the contribution. Talk about the results and finally talk about the limitations of the study.

% extra notes: could it be that the size of the company plays a role? Differential reaction to firms that have are large and profitable vs. to firms that are smaller and have lower profits. investors could perceive the actions of the smaller firm as unprofitable and therefore bad - Kim and Lyon (2015)

In times when climate change is a pressing issue, companies might want to utilize deceitful communications to enhance their financial performance. It is important to understand whether they can successfully employ such strategies, and what mechanisms are responsible for this behaviour.\footnote{I think I should try to expand on the limitations (in the context of future research) and on my results.}

This study contributes to this discussion, by presenting empirical results on impacts of greenwashing on financial performance of companies in presence of a positive sustainability demand shock in the form of climate activism. I find that high levels of greenwashing in companies' ESG reports may be met with decreases in performance of a company on the stock market, due to a negative reaction of investors. At the same time, the findings suggest that higher level of greenwashing may positively impact the quarterly sales growth of such companies. Nonetheless, both of these effects are very short-lived. 

Broadly, these results highlight a need for more standardized and transparent methods of corporate communication. As some of the greenwashing can be undetected by investors, and the sales growth increases in greenwashing, standardizing corporate communications could improve the relations between firms and their stakeholders. Implementing standardized reporting procedures could help align the incentives between all parties and thus achieve better outcomes. 

However, this study also has limitations that point to possible future research avenues. Firstly, this study has been severely limited by the data quality. Greenwashing is difficult to measure, and so, the results of this study could be affected by measurement error. Thus, more research in this area is needed. Secondly, this study could be made more efficient with the use of better empirical strategies. For example, it could include extra factors such as past volatility of a given stock in order to increase the precision of the findings. Moreover, future studies could account for differential effects based on the scale and profitability of a company, as these could imply differential effects. 

Thirdly, more research is needed in how different types of consumers differ in their perceptions and reactions to greenwashing. For example, in this study, a part of consumers could actually be businesses, who may react differently to an increase in climate activism and to greenwashing. Therefore, future research should account for differences across consumers groups. 

Finally, this study has employed assumptions that may not be fully satisfied, which implies that more research is required to establish whether the estimated results replicate in other similar scenarios. For example, it is needed to establish whether the way that a company advertises is correlated with the way that it reports to investors, to see whether the relation found in this study has theoretical foundations. 

Overall, this study offers empirical insight into greenwashing behaviour while acknowledging its contextual and methodological limitations. Consequently, these conclusions should be viewed as a useful starting point, rather than a definitive endpoint, and motivate future research.


% ---------------------------------------------------------------------------------------------------------
% ---------------------------------------------------------------------------------------------------------
% ---------------------------------------------------------------------------------------------------------

\pagebreak
\printbibliography{}
\pagebreak
\appendix

\section{Code}

The code needed for this study is publicly available \href{https://github.com/pszewi/thesis}{\underbar{here}}.

\renewcommand\thetable{\thesection.\arabic{table}}
\renewcommand\thefigure{\thesection.\arabic{figure}}
\setcounter{table}{0}
\setcounter{figure}{0}

\section{Additional Figures}\label{app:extra_figs}



This section presents additional figures related to the background section (Figure~\ref{fig:cc_trend}), as well as tables showing descriptive statistics differentiated by the treatment group.

\vfill{}
\centering

\begin{figure}[ht]

    \caption{Google Search Trends of ``Climate Change''}\label{fig:cc_trend}
    \includegraphics[width=\textwidth]{climchange_trend.pdf}
    \captionsetup{font=footnotesize}
    \caption*{Proportion of searches relative to the most searched, for the whole world. Red dots show major strikes organized my FFF.\@ These strikes happened on the following dates (from left to right): 15--03--2019, 24--05--2019, 20--09--2019, 27--09--2019, 29--11--2019. Source: Google Trends}
\end{figure}

\begin{table}

    \captionof{table}{Number of Reports per Year}\label{tab:sust_reps}
    \centering

    \begin{tabular}{cc}
        \toprule
        Year & Number of Reports\\
        \midrule
        2023 & 1859\\
        2022 & 2020\\
        2021 & 1733\\
        2020 & 1355\\
        2019 & 1081\\
        2018 & 852\\
        2017 & 686\\
        2016 & 558\\
        2015 & 475\\
        2014 & 415\\
        2013 & 357\\
        2012 & 300\\
        2011 & 252\\
        2010 & 216\\
        \bottomrule
    \end{tabular}

    \vspace{0.2cm}

    \begin{tablenotes}
        \footnotesize
        \item Number of responsibility reports retrieved per year. Source: \href{https://responsibilityreports.com}{ResponsibilityReports.com}
    \end{tablenotes}

\end{table}

% \vspace{5cm}

\begin{table}
    \captionof{table}{Estimation Periods for Each Variable}\label{tab:data_freq}
    \centering

    \begin{tabular}{l|ccc|ccc}
        \toprule
        Variable     & \multicolumn{3}{c|}{Normal Returns} & \multicolumn{3}{c}{Event Window} \\
                     & Start & End          & N                 &   Start & End     &     N \\ 
        \midrule
        Stock Prices & 2019--01--01 & 2019--07--31 & 211 & 2019--09--01 & 2019--10--10 & 28\\
        Sales        & ---           &     ---     & --- & 2018 Q3 & 2020 Q2 & 7\\  
        \bottomrule
    \end{tabular}

    \vspace{0.2cm}

    \begin{tablenotes}
        \footnotesize
        \item Date ranges for the pre-processing and estimation of each variable. Normal returns were only needed for the estimation of cumulative abnormal returns of stock prices.
    \end{tablenotes}
\end{table}


% \newpage

% ----------------------------------------------
\begin{table}
    \centering
    \captionof{table}{Allocation of Countries to Regional Markets}\label{tab:mrkt_ind}
    \footnotesize{
        \begin{tabular}{lc}
            \toprule
            Country & Market Index \\
            \midrule 
            United States & S\&P 500 Composite \\
            Canada & S\&P 500 Composite \\
            Bermuda & S\&P 500 Composite \\
        
            Mexico & MSCI EM Latin America  \\
            Panama & MSCI EM Latin America  \\
            Colombia & MSCI EM Latin America  \\
            Brazil & MSCI EM Latin America  \\
            Chile & MSCI EM Latin America  \\
            Peru & MSCI EM Latin America  \\
            Argentina & MSCI EM Latin America  \\
            
            United Kingdom & MSCI Europe  \\
            Ireland & MSCI Europe  \\
            Switzerland & MSCI Europe  \\
            Netherlands & MSCI Europe  \\
            Greece & MSCI Europe  \\
            Germany & MSCI Europe  \\
            Belgium & MSCI Europe  \\
            Denmark & MSCI Europe  \\
            Monaco & MSCI Europe  \\
            Luxembourg & MSCI Europe  \\
            France & MSCI Europe  \\
            Sweden & MSCI Europe  \\
            Spain & MSCI Europe  \\
            Finland & MSCI Europe  \\
            Guernsey & MSCI Europe \\ 
            Romania & MSCI Europe  \\
            Italy & MSCI Europe  \\
            
            
            Hong Kong & MSCI Pacific  \\
            Singapore & MSCI Pacific  \\
            Japan & MSCI Pacific  \\
            Australia & MSCI Pacific  \\
            New Zealand & MSCI Pacific  \\
            Papua New Guinea & MSCI Pacific  \\
            
            China & MSCI AC Asia  \\
            India & MSCI AC Asia  \\
            South Korea & MSCI AC Asia  \\
            Taiwan & MSCI AC Asia  \\
            Philippines & MSCI AC Asia  \\
            \bottomrule
        
        \end{tabular}
    }

\end{table}

% \newpage
% ----------------------------------------------

\begin{table}
    \centering
    \captionof{table}{Treatment Dose Distribution Across Countries (Stock data)}

    \footnotesize{
    \begin{tabular}{lccccc}
    \toprule
    Country & Overall & 1 & 2 & 3 & 4\\
    \midrule
    United States & 373 (59\%) & 123 (74\%) & 93 (59\%) & 88 (57\%) & 69 (45\%)\\
    United Kingdom & 64 (10\%) & 14 (8.4\%) & 13 (8.3\%) & 17 (11\%) & 20 (13\%)\\
    Canada & 55 (8.7\%) & 12 (7.2\%) & 11 (7.0\%) & 12 (7.8\%) & 20 (13\%)\\
    Australia & 47 (7.4\%) & 7 (4.2\%) & 10 (6.4\%) & 9 (5.8\%) & 21 (14\%)\\
    Ireland & 9 (1.4\%) & 0 (0\%) & 4 (2.5\%) & 4 (2.6\%) & 1 (0.6\%)\\
    Switzerland & 7 (1.1\%) & 1 (0.6\%) & 1 (0.6\%) & 2 (1.3\%) & 3 (1.9\%)\\
    Mexico & 6 (0.9\%) & 0 (0\%) & 2 (1.3\%) & 4 (2.6\%) & 0 (0\%)\\
    Bermuda & 6 (0.9\%) & 1 (0.6\%) & 4 (2.5\%) & 1 (0.6\%) & 0 (0\%)\\
    South Korea & 5 (0.8\%) & 0 (0\%) & 0 (0\%) & 1 (0.6\%) & 4 (2.6\%)\\
    Taiwan & 5 (0.8\%) & 0 (0\%) & 2 (1.3\%) & 2 (1.3\%) & 1 (0.6\%)\\
    India & 5 (0.8\%) & 1 (0.6\%) & 1 (0.6\%) & 2 (1.3\%) & 1 (0.6\%)\\
    Japan & 4 (0.6\%) & 0 (0\%) & 1 (0.6\%) & 1 (0.6\%) & 2 (1.3\%)\\
    Brazil & 4 (0.6\%) & 1 (0.6\%) & 1 (0.6\%) & 1 (0.6\%) & 1 (0.6\%)\\
    China & 4 (0.6\%) & 1 (0.6\%) & 2 (1.3\%) & 0 (0\%) & 1 (0.6\%)\\
    New Zealand & 4 (0.6\%) & 1 (0.6\%) & 2 (1.3\%) & 0 (0\%) & 1 (0.6\%)\\
    Luxembourg & 3 (0.5\%) & 2 (1.2\%) & 0 (0\%) & 1 (0.6\%) & 0 (0\%)\\
    Netherlands & 3 (0.5\%) & 0 (0\%) & 1 (0.6\%) & 1 (0.6\%) & 1 (0.6\%)\\
    Singapore & 3 (0.5\%) & 0 (0\%) & 0 (0\%) & 3 (1.9\%) & 0 (0\%)\\
    Argentina & 3 (0.5\%) & 1 (0.6\%) & 1 (0.6\%) & 0 (0\%) & 1 (0.6\%)\\
    Chile & 3 (0.5\%) & 1 (0.6\%) & 2 (1.3\%) & 0 (0\%) & 0 (0\%)\\
    Spain & 2 (0.3\%) & 0 (0\%) & 1 (0.6\%) & 0 (0\%) & 1 (0.6\%)\\
    Sweden & 2 (0.3\%) & 0 (0\%) & 1 (0.6\%) & 0 (0\%) & 1 (0.6\%)\\
    France & 2 (0.3\%) & 1 (0.6\%) & 1 (0.6\%) & 0 (0\%) & 0 (0\%)\\
    Greece & 2 (0.3\%) & 0 (0\%) & 0 (0\%) & 0 (0\%) & 2 (1.3\%)\\
    Hong Kong & 2 (0.3\%) & 0 (0\%) & 0 (0\%) & 1 (0.6\%) & 1 (0.6\%)\\
    Colombia & 1 (0.2\%) & 0 (0\%) & 0 (0\%) & 1 (0.6\%) & 0 (0\%)\\
    Finland & 1 (0.2\%) & 0 (0\%) & 0 (0\%) & 1 (0.6\%) & 0 (0\%)\\
    Germany & 1 (0.2\%) & 0 (0\%) & 0 (0\%) & 0 (0\%) & 1 (0.6\%)\\
    Guernsey & 1 (0.2\%) & 0 (0\%) & 0 (0\%) & 0 (0\%) & 1 (0.6\%)\\
    Italy & 1 (0.2\%) & 0 (0\%) & 1 (0.6\%) & 0 (0\%) & 0 (0\%)\\
    Panama & 1 (0.2\%) & 0 (0\%) & 0 (0\%) & 0 (0\%) & 1 (0.6\%)\\
    Papua New Guinea & 1 (0.2\%) & 0 (0\%) & 0 (0\%) & 1 (0.6\%) & 0 (0\%)\\
    Peru & 1 (0.2\%) & 0 (0\%) & 1 (0.6\%) & 0 (0\%) & 0 (0\%)\\
    Philippines & 1 (0.2\%) & 0 (0\%) & 0 (0\%) & 1 (0.6\%) & 0 (0\%)\\
    Romania & 1 (0.2\%) & 0 (0\%) & 1 (0.6\%) & 0 (0\%) & 0 (0\%)\\
    Unknown & 148 & 29 & 37 & 41 & 41\\
    \midrule
    Total & 781 & 196 &  194 & 195 & 196\\
    \bottomrule
    \end{tabular}

    \vspace{0.2cm}
    
    \begin{tablenotes}
        \footnotesize
        \item Number of firms per the country of domicile, differentiated by the treatment group. Source: Own calculation using \href{https://responsibilityreports.com}{ResponsibilityReports.com} and Refinitiv Database.
    \end{tablenotes}
    
    
    }
\end{table}

% \newpage

\begin{table}
    \centering
    \captionof{table}{Treatment Dose Distribution Across Countries (Sales data)}

        \begin{tabular}{lccccc}
        \toprule
        Country & Overall & 1 & 2 & 3 & 4 \\
        \midrule
        United States & 315 (77\%) & 105 (88\%) & 76 (76\%) & 79 (75\%) & 55 (65\%)\\
        Canada & 37 (9.0\%) & 6 (5.0\%) & 8 (8.0\%) & 9 (8.6\%) & 14 (16\%)\\
        United Kingdom & 6 (1.5\%) & 3 (2.5\%) & 1 (1.0\%) & 1 (1.0\%) & 1 (1.2\%)\\
        Ireland & 6 (1.5\%) & 0 (0\%) & 2 (2.0\%) & 3 (2.9\%) & 1 (1.2\%)\\
        Mexico & 5 (1.2\%) & 0 (0\%) & 1 (1.0\%) & 4 (3.8\%) & 0 (0\%)\\
        Switzerland & 4 (1.0\%) & 0 (0\%) & 0 (0\%) & 1 (1.0\%) & 3 (3.5\%)\\
        Bermuda & 4 (1.0\%) & 1 (0.8\%) & 3 (3.0\%) & 0 (0\%) & 0 (0\%)\\
        Brazil & 4 (1.0\%) & 1 (0.8\%) & 1 (1.0\%) & 1 (1.0\%) & 1 (1.2\%)\\
        Luxembourg & 3 (0.7\%) & 2 (1.7\%) & 0 (0\%) & 1 (1.0\%) & 0 (0\%)\\
        Japan & 3 (0.7\%) & 0 (0\%) & 1 (1.0\%) & 0 (0\%) & 2 (2.4\%)\\
        Chile & 2 (0.5\%) & 0 (0\%) & 2 (2.0\%) & 0 (0\%) & 0 (0\%)\\
        Greece & 2 (0.5\%) & 0 (0\%) & 0 (0\%) & 0 (0\%) & 2 (2.4\%)\\
        India & 2 (0.5\%) & 1 (0.8\%) & 0 (0\%) & 0 (0\%) & 1 (1.2\%)\\
        Singapore & 2 (0.5\%) & 0 (0\%) & 0 (0\%) & 2 (1.9\%) & 0 (0\%)\\
        Sweden & 2 (0.5\%) & 0 (0\%) & 1 (1.0\%) & 0 (0\%) & 1 (1.2\%)\\
        Taiwan & 2 (0.5\%) & 0 (0\%) & 1 (1.0\%) & 1 (1.0\%) & 0 (0\%)\\
        Israel & 1 (0.2\%) & 0 (0\%) & 0 (0\%) & 1 (1.0\%) & 0 (0\%)\\
        Argentina & 1 (0.2\%) & 0 (0\%) & 1 (1.0\%) & 0 (0\%) & 0 (0\%)\\
        China & 1 (0.2\%) & 1 (0.8\%) & 0 (0\%) & 0 (0\%) & 0 (0\%)\\
        Finland & 1 (0.2\%) & 0 (0\%) & 0 (0\%) & 1 (1.0\%) & 0 (0\%)\\
        Germany & 1 (0.2\%) & 0 (0\%) & 0 (0\%) & 0 (0\%) & 1 (1.2\%)\\
        Netherlands & 1 (0.2\%) & 0 (0\%) & 0 (0\%) & 0 (0\%) & 1 (1.2\%)\\
        Panama & 1 (0.2\%) & 0 (0\%) & 0 (0\%) & 0 (0\%) & 1 (1.2\%)\\
        Peru & 1 (0.2\%) & 0 (0\%) & 1 (1.0\%) & 0 (0\%) & 0 (0\%)\\
        Philippines & 1 (0.2\%) & 0 (0\%) & 0 (0\%) & 1 (1.0\%) & 0 (0\%)\\
        Spain & 1 (0.2\%) & 0 (0\%) & 1 (1.0\%) & 0 (0\%) & 0 (0\%)\\
        United Arab Emirates & 1 (0.2\%) & 0 (0\%) & 0 (0\%) & 0 (0\%) & 1 (1.2\%)\\
        Unknown & 93 & 18 & 24 & 27 & 24\\
        \midrule
        Total & 503 & 138 &  124 & 132 & 109 \\
        \bottomrule
    \end{tabular}

    \vspace{0.2cm}
    
    \begin{tablenotes}
        \footnotesize
        \item Number of firms per the country of domicile, differentiated by the treatment group. Source: Own calculation using \href{https://responsibilityreports.com}{ResponsibilityReports.com} and Refinitiv Database.
    \end{tablenotes}
    
    

\end{table}

% \newpage

% ----------------------------------------------
\begin{table}
    \centering
    \captionof{table}{Treatment Distribution Across Stock Exchanges (Stock data)}\label{tab:exchngs}
    
    \begin{tabular}{lccccc}
        \toprule
        Exchange & Overall & 1  & 2 & 3 & 4 \\
        \midrule
        NYSE & 379 (49\%) & 109 (56\%) & 99 (51\%) & 98 (50\%) & 73 (37\%)\\
        Nasdaq & 100 (13\%) & 32 (16\%) & 22 (11\%) & 24 (12\%) & 22 (11\%)\\
        London SE & 68 (8.7\%) & 13 (6.6\%) & 17 (8.8\%) & 18 (9.2\%) & 20 (10\%)\\
        Toronto SE & 56 (7.2\%) & 17 (8.7\%) & 9 (4.6\%) & 12 (6.2\%) & 18 (9.2\%)\\
        Australian SE & 55 (7.0\%) & 8 (4.1\%) & 13 (6.7\%) & 9 (4.6\%) & 25 (13\%)\\
        Boerse Frankfurt & 23 (2.9\%) & 5 (2.6\%) & 5 (2.6\%) & 7 (3.6\%) & 6 (3.1\%)\\
        Tokyo SE & 20 (2.6\%) & 0 (0\%) & 5 (2.6\%) & 7 (3.6\%) & 8 (4.1\%)\\
        Euronext Paris & 12 (1.5\%) & 0 (0\%) & 3 (1.5\%) & 3 (1.5\%) & 6 (3.1\%)\\
        Hong Kong Exchange & 10 (1.3\%) & 2 (1.0\%) & 5 (2.6\%) & 1 (0.5\%) & 2 (1.0\%)\\
        Borsa Italiana & 7 (0.9\%) & 1 (0.5\%) & 2 (1.0\%) & 2 (1.0\%) & 2 (1.0\%)\\
        NASDAQ Helsinki & 7 (0.9\%) & 3 (1.5\%) & 0 (0\%) & 2 (1.0\%) & 2 (1.0\%)\\
        NASDAQ Stockholm & 6 (0.8\%) & 1 (0.5\%) & 1 (0.5\%) & 1 (0.5\%) & 3 (1.5\%)\\
        Six Swiss Exchange & 6 (0.8\%) & 1 (0.5\%) & 2 (1.0\%) & 1 (0.5\%) & 2 (1.0\%)\\
        NASDAQ Copenhagen & 4 (0.5\%) & 0 (0\%) & 1 (0.5\%) & 0 (0\%) & 3 (1.5\%)\\
        BME Exchange & 4 (0.5\%) & 0 (0\%) & 1 (0.5\%) & 2 (1.0\%) & 1 (0.5\%)\\
        Euronext Amsterdam & 4 (0.5\%) & 1 (0.5\%) & 1 (0.5\%) & 1 (0.5\%) & 1 (0.5\%)\\
        Singapore Exchange & 3 (0.4\%) & 1 (0.5\%) & 0 (0\%) & 2 (1.0\%) & 0 (0\%)\\
        Wiener Boerse AG & 3 (0.4\%) & 0 (0\%) & 2 (1.0\%) & 1 (0.5\%) & 0 (0\%)\\
        Euronext Brussels & 3 (0.4\%) & 0 (0\%) & 1 (0.5\%) & 2 (1.0\%) & 0 (0\%)\\
        Taiwan SE & 2 (0.3\%) & 1 (0.5\%) & 1 (0.5\%) & 0 (0\%) & 0 (0\%)\\
        Korea Exchange & 2 (0.3\%) & 1 (0.5\%) & 0 (0\%) & 1 (0.5\%) & 0 (0\%)\\
        National SE & 2 (0.3\%) & 0 (0\%) & 2 (1.0\%) & 0 (0\%) & 0 (0\%)\\
        Oslo Bors & 2 (0.3\%) & 0 (0\%) & 1 (0.5\%) & 0 (0\%) & 1 (0.5\%)\\
        Athens SE & 1 (0.1\%) & 0 (0\%) & 0 (0\%) & 1 (0.5\%) & 0 (0\%)\\
        Boerse Hamburg & 1 (0.1\%) & 0 (0\%) & 0 (0\%) & 0 (0\%) & 1 (0.5\%)\\
        Santiago SE & 1 (0.1\%) & 0 (0\%) & 1 (0.5\%) & 0 (0\%) & 0 (0\%)\\
        \midrule
        Total &  781 & 196 & 194 & 195 & 196 \\
        \bottomrule
    \end{tabular}

    \vspace{0.2cm}

    \begin{tablenotes}
        \footnotesize
        \item Number of firms per the stock exchange of listing, differentiated by the treatment group. Source: Own calculation using \href{https://responsibilityreports.com}{ResponsibilityReports.com} and Refinitiv Database.
    \end{tablenotes}

\end{table}


% ----------------------------------------------
% \newpage
\begin{table}
    \centering
    \captionof{table}{Treatment Distribution Across Stock Exchanges (Sales data)}\label{tab:exchngs_sale}
    
    \begin{tabular}{lccccc}
        \toprule
        Exchange & Overall & 1 & 2 & 3 & 4 \\
        \midrule
        NYSE & 299 (59\%) & 90 (65\%) & 77 (62\%) & 78 (59\%) & 54 (50\%)\\
        Nasdaq & 86 (17\%) & 27 (20\%) & 17 (14\%) & 22 (17\%) & 20 (18\%)\\
        Toronto SE & 37 (7.4\%) & 10 (7.2\%) & 7 (5.6\%) & 8 (6.1\%) & 12 (11\%)\\
        Tokyo SE & 18 (3.6\%) & 0 (0\%) & 4 (3.2\%) & 7 (5.3\%) & 7 (6.4\%)\\
        Boerse Frankfurt & 17 (3.4\%) & 3 (2.2\%) & 4 (3.2\%) & 7 (5.3\%) & 3 (2.8\%)\\
        Euronext Paris & 10 (2.0\%) & 0 (0\%) & 3 (2.4\%) & 3 (2.3\%) & 4 (3.7\%)\\
        NASDAQ Helsinki & 6 (1.2\%) & 3 (2.2\%) & 0 (0\%) & 2 (1.5\%) & 1 (0.9\%)\\
        NASDAQ Stockholm & 6 (1.2\%) & 1 (0.7\%) & 1 (0.8\%) & 1 (0.8\%) & 3 (2.8\%)\\
        NASDAQ Copenhagen & 3 (0.6\%) & 0 (0\%) & 1 (0.8\%) & 0 (0\%) & 2 (1.8\%)\\
        Wiener Boerse AG & 3 (0.6\%) & 0 (0\%) & 2 (1.6\%) & 1 (0.8\%) & 0 (0\%)\\
        BME Exchange & 2 (0.4\%) & 0 (0\%) & 1 (0.8\%) & 1 (0.8\%) & 0 (0\%)\\
        Borsa Italiana & 2 (0.4\%) & 1 (0.7\%) & 1 (0.8\%) & 0 (0\%) & 0 (0\%)\\
        Euronext Amsterdam & 2 (0.4\%) & 0 (0\%) & 1 (0.8\%) & 1 (0.8\%) & 0 (0\%)\\
        Euronext Brussels & 2 (0.4\%) & 0 (0\%) & 1 (0.8\%) & 1 (0.8\%) & 0 (0\%)\\
        London SE & 2 (0.4\%) & 1 (0.7\%) & 0 (0\%) & 0 (0\%) & 1 (0.9\%)\\
        Six Swiss Exchange & 2 (0.4\%) & 1 (0.7\%) & 1 (0.8\%) & 0 (0\%) & 0 (0\%)\\
        Taiwan SE & 2 (0.4\%) & 1 (0.7\%) & 1 (0.8\%) & 0 (0\%) & 0 (0\%)\\
        Boerse Hamburg & 1 (0.2\%) & 0 (0\%) & 0 (0\%) & 0 (0\%) & 1 (0.9\%)\\
        Hong Kong Exchange & 1 (0.2\%) & 0 (0\%) & 1 (0.8\%) & 0 (0\%) & 0 (0\%)\\
        National SE & 1 (0.2\%) & 0 (0\%) & 1 (0.8\%) & 0 (0\%) & 0 (0\%)\\
        Oslo Bors & 1 (0.2\%) & 0 (0\%) & 0 (0\%) & 0 (0\%) & 1 (0.9\%)\\
        \midrule
        Total & 503 & 138 & 124 & 132 & 109\\
        \bottomrule
    \end{tabular}

    \vspace{0.2cm}

    \begin{tablenotes}
        \footnotesize
        \item Number of firms per the stock exchange of listing, differentiated by the treatment group. Source: Own calculation using \href{https://responsibilityreports.com}{ResponsibilityReports.com} and Refinitiv Database.
    \end{tablenotes}

\end{table}


% ----------------------------------------------
% \newpage




\begin{table}
    \centering
    \captionof{table}{Treatment Distribution Across Industries}\label{tab:industries}

    (a) Stock data
    \vspace{0.5cm}

    \begin{tabular}{lccccc}
    \toprule
    Industry & Overall & 1 & 2 & 3 & 4 \\
    \midrule
    Industrials & 119 (19\%) & 31 (19\%) & 33 (21\%) & 27 (18\%) & 28 (18\%)\\
    Consumer Discretionary & 99 (16\%) & 39 (23\%) & 32 (20\%) & 21 (14\%) & 7 (4.5\%)\\
    Financials & 87 (14\%) & 11 (6.6\%) & 13 (8.3\%) & 20 (13\%) & 43 (28\%)\\
    Basic Materials & 82 (13\%) & 17 (10\%) & 17 (11\%) & 25 (16\%) & 23 (15\%)\\
    Consumer Staples & 56 (8.8\%) & 19 (11\%) & 17 (11\%) & 14 (9.1\%) & 6 (3.9\%)\\
    Technology & 52 (8.2\%) & 5 (3.0\%) & 11 (7.0\%) & 18 (12\%) & 18 (12\%)\\
    Utilities & 46 (7.3\%) & 21 (13\%) & 15 (9.6\%) & 9 (5.8\%) & 1 (0.6\%)\\
    Health Care & 45 (7.1\%) & 13 (7.8\%) & 5 (3.2\%) & 11 (7.1\%) & 16 (10\%)\\
    Telecommunications & 19 (3.0\%) & 2 (1.2\%) & 4 (2.5\%) & 6 (3.9\%) & 7 (4.5\%)\\
    Energy & 16 (2.5\%) & 6 (3.6\%) & 4 (2.5\%) & 1 (0.6\%) & 5 (3.2\%)\\
    Real Estate & 12 (1.9\%) & 3 (1.8\%) & 6 (3.8\%) & 2 (1.3\%) & 1 (0.6\%)\\
    Unknown & 148 & 29 & 37 & 41 & 41\\
    \midrule 
    Total & 781 & 196 & 194 & 195 & 196\\
    \bottomrule
    \end{tabular}

    \vspace{1cm}
    (b) Sales data
    \vspace{0.5cm}

    \begin{tabular}{lccccc}
        \toprule
        Industry & Overall & 1 & 2 & 3 & 4 \\
        \midrule
        Industrials & 79 (19\%) & 27 (23\%) & 20 (20\%) & 19 (18\%) & 13 (15\%)\\
        Consumer Discretionary & 58 (14\%) & 27 (23\%) & 16 (16\%) & 14 (13\%) & 1 (1.2\%)\\
        Financials & 49 (12\%) & 4 (3.3\%) & 7 (7.0\%) & 13 (12\%) & 25 (29\%)\\
        Basic Materials & 45 (11\%) & 9 (7.5\%) & 12 (12\%) & 13 (12\%) & 11 (13\%)\\
        Technology & 40 (9.8\%) & 5 (4.2\%) & 8 (8.0\%) & 14 (13\%) & 13 (15\%)\\
        Consumer Staples & 38 (9.3\%) & 13 (11\%) & 13 (13\%) & 10 (9.5\%) & 2 (2.4\%)\\
        Health Care & 35 (8.5\%) & 13 (11\%) & 4 (4.0\%) & 8 (7.6\%) & 10 (12\%)\\
        Utilities & 31 (7.6\%) & 14 (12\%) & 10 (10\%) & 6 (5.7\%) & 1 (1.2\%)\\
        Telecommunications & 15 (3.7\%) & 2 (1.7\%) & 3 (3.0\%) & 6 (5.7\%) & 4 (4.7\%)\\
        Energy & 12 (2.9\%) & 4 (3.3\%) & 4 (4.0\%) & 0 (0\%) & 4 (4.7\%)\\
        Real Estate & 8 (2.0\%) & 2 (1.7\%) & 3 (3.0\%) & 2 (1.9\%) & 1 (1.2\%)\\
        Unknown & 93 & 18 & 24 & 27 & 24\\
        \midrule
        Total & 503 & 138 & 124 & 132 & 109 \\
        \bottomrule
    \end{tabular}

    \vspace{0.2cm}

    \begin{tablenotes}
        \footnotesize
        \item Number of firms per industry, differentiated by the treatment group. Source: Own calculation using \href{https://responsibilityreports.com}{ResponsibilityReports.com} and Refinitiv Database.
    \end{tablenotes}

\end{table}

% ----------------------------------------------


% ----------------------------------------------

\begin{table}
    \centering
    \captionof{table}{Treatment Distribution Across Firm Sizes}\label{tab:firm_size}
    
    (a) Stock data
    \vspace{0.5cm}

    \begin{tabular}{lccccc}
        \toprule
        Number of Employees & Overall & 1 & 2 & 3 & 4 \\
        \midrule
        10,000+ Employees & 478 (62\%) & 117 (60\%) & 116 (60\%) & 128 (66\%) & 117 (61\%)\\
        5001--10,000 Employees & 118 (15\%) & 36 (18\%) & 29 (15\%) & 32 (16\%) & 21 (11\%)\\
        1001--5000 Employees & 132 (17\%) & 34 (17\%) & 34 (18\%) & 25 (13\%) & 39 (20\%)\\
        501--1000 Employees & 13 (1.7\%) & 3 (1.5\%) & 4 (2.1\%) & 2 (1.0\%) & 4 (2.1\%)\\
        201--500 Employees & 20 (2.6\%) & 3 (1.5\%) & 7 (3.6\%) & 5 (2.6\%) & 5 (2.6\%)\\
        51--200 Employees & 6 (0.8\%) & 0 (0\%) & 2 (1.0\%) & 1 (0.5\%) & 3 (1.6\%)\\
        11--50 Employees & 7 (0.9\%) & 2 (1.0\%) & 1 (0.5\%) & 1 (0.5\%) & 3 (1.6\%)\\
        1--10 Employees & 1 (0.1\%) & 0 (0\%) & 0 (0\%) & 0 (0\%) & 1 (0.5\%)\\
        Unknown & 6 & 1 & 1 & 1 & 3\\
        \midrule
        Total &  781 & 196 & 194 & 195 & 196 \\
        \bottomrule
    \end{tabular}
    
    \vspace{1cm}
    (b) Sales data
    \vspace{0.5cm}

    \begin{tabular}{lccccc}
        \toprule
        Number of Employees & Overall & 1 & 2 & 3 & 4\\
        \midrule
        10,000+ Employees & 338 (68\%) & 93 (68\%) & 75 (61\%) & 95 (72\%) & 75 (69\%)\\
        5001--10,000 Employees & 78 (16\%) & 21 (15\%) & 21 (17\%) & 19 (14\%) & 17 (16\%)\\
        1001--5000 Employees & 66 (13\%) & 19 (14\%) & 21 (17\%) & 14 (11\%) & 12 (11\%)\\
        501--1000 Employees & 7 (1.4\%) & 2 (1.5\%) & 2 (1.6\%) & 1 (0.8\%) & 2 (1.9\%)\\
        201--500 Employees & 8 (1.6\%) & 2 (1.5\%) & 2 (1.6\%) & 2 (1.5\%) & 2 (1.9\%)\\
        51--200 Employees & 2 (0.4\%) & 0 (0\%) & 1 (0.8\%) & 1 (0.8\%) & 0 (0\%)\\
        11--50 Employees & 1 (0.2\%) & 0 (0\%) & 1 (0.8\%) & 0 (0\%) & 0 (0\%)\\
        Unknown & 3 & 1 & 1 & 0 & 1\\
        \midrule
        Total & 503 & 138 & 124 & 132  & 109\\
        \bottomrule
    \end{tabular}


    \vspace{0.2cm}

    \begin{tablenotes}
        \footnotesize
        \item Number of firms per the number of employees, differentiated by the treatment group. Source: Own calculation using \href{https://responsibilityreports.com}{ResponsibilityReports.com} and Refinitiv Database.
    \end{tablenotes}
\end{table}


\newpage

\normalsize
\raggedright{}
\setcounter{table}{0}
\setcounter{figure}{0}


\section{Data}\label{app:data}

This section provides additional descriptions to the process of creating the dataset used in the study.

\subsection{Cheap-Talk Index}\label{app:data:cti}

The models for estimation were retrieved from\ \dots

The table below presents multiple examples of data classified by the researchers in training the model:\ \dots 

Reports that contained less than 3 climate related paragraphs were dropped due to having to little data, which was prone to producing outliers.

\subsection{Merging}\label{app:data:merging}

The merging of the data was a difficult process, as company names are often reported differently between different services. Multiple strategies were used to merge the data. Firstly, the data was merged on the names retrieved from the \href{https://responsibilityreports.com}{ResponsibilityReports.com} and the longest version of names available in the Refinitiv database.\@ \textbf{This resulted in \dots\ observations}. After that, the datasets were merged on the stock tickers conditional on being listed on the same exchange.\@ \textbf{This resulted in \dots\ observations}. Finally, the remaining companies were merged using fuzzy matching.\ \textbf{Unfortunately this has left \dots\ observations missing}.

\section{Models}\label{app:models}

This section presents the formal versions of the model assumptions using the framework of potential outcomes of~\textcite{rubinEstimatingCausalEffects1974}.

\subsection{Default Difference-in-Differences}
\subsubsection{Parallel Trends (PT)}

Formally it is given by: 
\begin{equation}
    \mathbb{E}[Y_{i,2}(0) - Y_{i,1}(0) | D_i = 1] = \mathbb{E}[Y_{i,2}(0) - Y_{i,1}(0) | D_i = 0]
\end{equation}

\subsubsection{No anticipation}
Formally, the assumption is defined as:

\begin{equation}
    Y_{i,1}(0) = Y_{i,1}(1) \text{ for all } i \text{ with } D_i = 1
\end{equation}


\subsection{Multivalued Difference-in-Differences}

% \small
% \centering

\renewcommand\thefigure{\thesection.\arabic{figure}}
\section{Robustness Checks}\label{app:robustness}
\setcounter{figure}{0} 

This section presents event studies for robustness checks of the no-anticipation assumption (Section~\ref{subsect:stocks}). Figure~\ref{fig:eve_stock_rbc_6} shows the event study with treatment period being moved up to 5 days before the first climate strike (15th of September). The reference day is set as the last day before the weekend (13th of September). Figure~\ref{fig:eve_stock_rbc_10} shows the event study with treatment period being moved up to 10 days before the first climate strike. The reference day is set to 9th of September. As visible from both of the figures, there are no anticipatory effects visible in the original pre-treatment period.

\begin{figure}
    \caption{Stock Prices Robustness Check --- 5 days}\label{fig:eve_stock_rbc_6}
    \centering
    
    (a) Group 2
    
    \includegraphics[width=0.79\textwidth]{stock_price_eve_2_rbc_6.pdf} \\
    
    (b) Group 3
    
    \includegraphics[width=0.79\textwidth]{stock_price_eve_3_rbc_6.pdf} \\
    
    (c) Group 4
    
    \includegraphics[width=0.79\textwidth]{stock_price_eve_4_rbc_6.pdf}
    
    \captionsetup{font=footnotesize}
    \caption*{The effect of receiving treatment on the cumulative abnormal returns of stock prices. Group number indicates the treatment dose, with 2 being the lowest and 4 being the highest. Gaps in the graph are due to the inclusion of weekends in the data. 15th of September 2019 marks the first period of treatment and 13th of September 2019 is set as the reference. 95\% confidence intervals are reported.}
\end{figure}

\begin{figure}
    \caption{Stock Prices Robustness Check --- 10 days}\label{fig:eve_stock_rbc_10}
    \centering
    
    (a) Group 2
    
    \includegraphics[width=0.79\textwidth]{stock_price_eve_2_rbc_10.pdf} \\
    
    (b) Group 3
    
    \includegraphics[width=0.79\textwidth]{stock_price_eve_3_rbc_10.pdf} \\
    
    (c) Group 4
    
    \includegraphics[width=0.79\textwidth]{stock_price_eve_4_rbc_10.pdf}
    
    \captionsetup{font=footnotesize}
    \caption*{The effect of receiving treatment on the cumulative abnormal returns of stock prices. Group number indicates the treatment dose, with 2 being the lowest and 4 being the highest. Gaps in the graph are due to the inclusion of weekends in the data. 10th of September 2019 marks the first period of treatment and 9th of September 2019 is set as the reference. 95\% confidence intervals are reported.}
\end{figure}


\end{document}